\begin{task}[credit=16]{Entscheidungsbäume - ID3 Algorithmus}
Die folgende Tabelle zeigt die Entscheidung, ob Baseball gespielt wird, basierend auf vier Wetterattributen.

\begin{table}[h]
\centering
\caption{Trainingsdatensatz, ob Baseball gespielt wird basierend auf der Wetterlage.}
\label{tab:data_baseball}
\begin{tabular}{l|c|c|c|c|c}
\toprule
\textbf{Tag} & \textbf{Ausblick (A)} & \textbf{Temperatur (T)}  & \textbf{Luftfeuchtigkeit (L)} & \textbf{Wind (W)}     & \textbf{Spielt Baseball (B)} \\
\midrule
T1  & Sonnig    & Warm        & Hoch             & Schwach  & Nein            \\
T2  & Sonnig    & Warm        & Hoch             & Stark    & Nein            \\
T3  & Bewölkung & Warm        & Hoch             & Schwach  & Ja              \\
T4  & Regen     & Mild        & Hoch             & Schwach  & Ja              \\
T5  & Regen     & Kühl        & Normal           & Schwach  & Ja              \\
T6  & Regen     & Kühl        & Normal           & Stark    & Nein            \\
T7  & Bewölkung & Kühl        & Normal           & Stark    & Ja              \\
T8  & Sonnig    & Mild        & Hoch             & Schwach  & Nein            \\
T9  & Sonnig    & Kühl        & Normal           & Schwach  & Ja              \\
T10 & Regen     & Mild        & Normal           & Schwach  & Ja              \\
T11 & Sonnig    & Mild        & Normal           & Stark    & Ja              \\
T12 & Bewölkung & Mild        & Hoch             & Stark    & Ja              \\
T13 & Bewölkung & Warm        & Normal           & Schwach  & Ja              \\
T14 & Regen     & Mild        & Hoch             & Stark    & Nein            \\
\bottomrule
\end{tabular}
\end{table}

\begin{table}[h!]
\centering
\caption{Vorhersage-Datensatz, ob Baseball gespielt wird.}
\label{tab:data_baseball_predict}
\begin{tabular}{l|c|c|c|c|c}
\toprule
\textbf{Tag} & \textbf{Ausblick (A)} & \textbf{Temperatur (T)}  & \textbf{Luftfeuchtigkeit (L)} & \textbf{Wind (W)}     & \textbf{Spielt Baseball (B)} \\
\midrule
T15 & Sonnig    & Mild        & Hoch     & Schwach & ?       \\
T16 & Bewölkung & Mild        & Normal   & Schwach & ?       \\
T17 & Regen     & Kühl        & Normal   & Stark   & ?       \\
\bottomrule
\end{tabular}
\end{table}

Die Aufgabe ist es folgende Frage zu beantworten: \textit{Unter welchen Bedingungen wir Baseball gespielt?}

\begin{subtask}[points=10,title=ID3 Algorithmus]
\label{q:id3_alg}
Erstellen Sie den Entscheidungsbaum mittels des ID3 Algorithmus.
Berechnen Sie dabei die \textbf{Entropie} und den \textbf{Informationsgewinn} (engl. \textit{gain}) der Attribut-Selektion für jeden Schritt.
Verwenden Sie bei der Berechnung der Entropie den Logarithmus zur Basis 2, Logarithmus-Dualis.

\textbf{Hinweis:} Sie können \textbf{Spielt Baseball (B)} mit $B$ kennzeichnen. Der Informationsgewinn ist nach Vorlesung wie folgt definiert:
\textit{Differenz zwischen den Informationen der Beispiele mit und ohne die Aufteilung durch $X_j$.}

Bsp. Informationsgewinn für Aufteilung des Wurzelknotens nach dem Merkmal \textit{Ausblick}. 
\begin{align*}
Gain\left (B, Ausblick \right ) &= Entropy\left (B \right ) \\&- \frac{B_\text{Ausblick=Sonnig}}{B}\cdot Entropy\left (B_\text{Ausblick=Sonnig} \right ) \\
&- \frac{B_\text{Ausblick=Regen}}{B}\cdot Entropy\left (B_\text{Ausblick=Regen} \right )\\ &- \frac{B_\text{Ausblick=Bewölkung}}{B}\cdot Entropy\left (B_\text{Ausblick=Bewölkung} \right )
\end{align*}

\begin{solution}
% Geben Sie hier Ihre Antwort an.
\end{solution}
\end{subtask}

\begin{subtask}[points=3,title=Visualisierung]
Erstellen Sie eine Visualsierung (Plot oder eingefügte Zeichnung) des Entscheidungsbaumes aus Aufgabenteil~\ref{q:id3_alg}.

 \begin{solution}
% Geben Sie hier Ihre Antwort an.
\end{solution}

\end{subtask}

\begin{subtask}[points=3,title=Vorhersage]
Geben Sie anhand ihres Entscheidungsbaumes eine Vorhersage für die Tage 15 bis 17 aus Tabelle~\ref{tab:data_baseball_predict}, ob Baseball gespielt wird.
\begin{solution}
% Geben Sie hier Ihre Antwort an.
\end{solution}
\end{subtask}
\end{task}