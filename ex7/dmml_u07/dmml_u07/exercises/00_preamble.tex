\centering{Diese Übung wird am \textbf{01.07.2021} um \textbf{13:30 Uhr} besprochen und \textbf{nicht bewertet}.}

\begin{table}[h!]
\centering
\begin{tabular}{c|c|c|c|c|c|c|c|c}
\toprule
\textbf{Aufgabe}              & 1  & 2  & 3  & 4 & 5 \\ \hline
\textbf{Maximal Punktzahl}    & 16 & 15 & 19 & 6 & 7  \\ \hline
\textbf{Erreichte Punktzahl}  &   &   &   &   &    \\
\bottomrule
\end{tabular}
\end{table}

\flushleft

\par \textbf{Benötigte Dateien}\\
Alle benötigten Datensätze und Skriptvorlagen finden Sie in unserem Moodle-Kurs: \urlc{https://moodle.informatik.tu-darmstadt.de/course/view.php?id=1058}

\textbf{Theoretische Aufgaben}\\
Bei diesen Übungsaufgaben können Sie Ihren Lösungsvorschlag in \LaTeX~formatieren.
Nutzen Sie hierfür die \LaTeX-Vorlage und die vorgesehene Blöcke.
\begin{verbatim}
\begin{solution}
% Geben Sie hier Ihre Antwort an.
\end{solution}
\end{verbatim}
